
\documentclass[a4paper,12pt]{article}
\usepackage{geometry}
\geometry{margin=1in}
\begin{document}


%%%%%%%%%%%%%%%%%%%%%%%%%%%%%%%%%%%%%%%%%%%%%%%%%%%%%%%%%%%%%%%%%%%%%%%%%%%%%%% TITLEPAGE
\pagenumbering{gobble}
\centering
\vspace*{5cm}
{\huge Developing AntBot: \par Visual Navigation based on the insect brain \par}
\vspace{1cm}
{\itshape Robert Mitchell}

\vspace{2cm}

{\large Master of Informatics \par}
{\large Informatics \par}
{\large School of Informatics \par}
{\large The University of Edinburgh \par}
\large \today \par

\vfill
Supervised by\par
Dr. Barbara Webb

\newpage

%%%%%%%%%%%%%%%%%%%%%%%%%%%%%%%%%%%%%%%%%%%%%%%%%%%%%%%%%%%%%%%%%%%%%%%%%%%%%%% ACKNOWLEDGEMENTS
\pagenumbering{roman}
\centering
{\LARGE \textbf{Acknowledgements}}
\begin{flushleft}
 {\small
  I would like to take the opportunity to thank my supervisor, Dr. Barbara Webb, for her
  guidance, and invaluable insight on the subject matter. My gratitude also extends to
  Zhaoyu Zhang and Leonard Eberding, two of my predecessors on this project; both have been
  extremely useful in explaining the existing codebase and operations of the robot where they
  were not always clear. Finally I should like to thank my parents, for their unwavering support
  throughout my education; I could not have made it here without them. }
\end{flushleft}  

\newpage
%%%%%%%%%%%%%%%%%%%%%%%%%%%%%%%%%%%%%%%%%%%%%%%%%%%%%%%%%%%%%%%%%%%%%%%%%%%%%%% ABSTRACT
\centering
{\LARGE\textbf{Abstract}}
\begin{flushleft}
{\small }
\end{flushleft}

\newpage

%%%%%%%%%%%%%%%%%%%%%%%%%%%%%%%%%%%%%%%%%%%%%%%%%%%%%%%%%%%%%%%%%%%%%%%%%%%%%%% DECLARATION
\centering
{\LARGE\textbf{Declaration}}
\begin{flushleft}
  {\small
    I declare that this disseration was composed by myself, the work
    contained herein is my own except where explicitly stated otherwise
    in the text, and that this work has not been submitted for any other
    degree or professional qualification except as specified.
    \par 

    \textit{Robert Mitchell}}

\end{flushleft}

\newpage
%%%%%%%%%%%%%%%%%%%%%%%%%%%%%%%%%%%%%%%%%%%%%%%%%%%%%%%%%%%%%%%%%%%%%%%%%%%%%%% CONTENTS


\tableofcontents

\newpage

%%%%%%%%%%%%%%%%%%%%%%%%%%%%%%%%%%%%%%%%%%%%%%%%%%%%%%%%%%%%%%%%%%%%%%%%%%%%%%% FIGURES
\listoffigures
\newpage

%%%%%%%%%%%%%%%%%%%%%%%%%%%%%%%%%%%%%%%%%%%%%%%%%%%%%%%%%%%%%%%%%%%%%%%%%%%%%%% TABLES
\listoftables
\newpage
%%%%%%%%%%%%%%%%%%%%%%%%%%%%%%%%%%%%%%%%%%%%%%%%%%%%%%%%%%%%%%%%%%%%%%%%%%%%%%% BODY
%%%%%%%%%%%%%%%%%%%%%%%%%%%%%%%%%%%%%%%%%%%%%%%%%%%%%%%%%%%%%%%%%%%%%%%%%%%%%%% INTRODUCTION
\pagenumbering{arabic}

\raggedright
\section{ Introduction }
Desert ants (\textit{Cataglyphis velox}) have the
remarkable ability to navigate through complex natural environments, using only low-resolution
visual information and limited computational power. It is well documented that many species of
ant, and other hymenoptra are capable of very robust visual navigation; however, it is as yet
unclear how the insects perform this seemingly complex task with such little brainpower. In this
paper, we will focus on using and extending an existing model for visual navigation in ants using
the Mushroom Body circuit, an artificial neural network which emulates the Mushroom Body neuropils
in the ant brain. We will also discuss biologically plausible methods of visual Collision Avoidance
using Optical Flow. A robot (AntBot) has been constructed \cite{Eberding2015} to allow us a testing
platform on which to implement, and experiment with, the algorithms in the
\textit{Ant Navigational Toolkit} \cite{Wehner2009}.

\subsection{ Motivation }
Though we are able to observe, and mimic algorithmically, the visual navigational capabilities
of insects, we still do not understand the precise methods by which this process takes place. The
model we will look at was propsed by \textit{Ardin et al.} \cite{Ardin2016}, which takes
the Mushroom Body (whose function was thought to be primarily for olfactory learning), and shows
that this provides a plausible neural model for encoding visual memories.
\newline

The MB circuit has been implemented and tested on AntBot by Eberding and Zhang respectively, however
the existing MB circuit is fairly simple. It uses binary weightings for the connections between
the visual projection neurons and the Kenyon Cells, and a single boolean Extrinsic Neuron denoting
image recognition. A modification was made by Zhang, whereby eight ENs were used, one for each of
the cardinal directions in the Central Complex model. This will be discussed further in
\ref{MBBackground}. The reader should note that the Central Complex (CX) model is primarily used
to model the task of Path Integration and will not be discussed further (see \cite{Scimeca2016}).
\newline

We would also like to look at methods for collision avoidance (CA) which do not involve specialised
sensors such as a LIDAR or SONAR, the luxury of which, ants do not have. Models have been proposed
which use Optical Flow (OF) properties to determine whether or not a collision is imminent. These
models have been propsed both in a purely robotic context \cite{Souhila2007},
and biological ones \cite{Low2005, Stewart2010}. 
  
\subsection { Goals }
% EDIT: The reverse route may not be worth talking about, or perhaps integrated path integration?
% I'm not sure there is enough time for this

The project aims for the following experimental scenario to be possible: We want to send the robot
on a run through an obstacle course, allowing it to navigate however it chooses through the
environment. From here, we want the robot to be able to replicate this route using only visual
memories, which it should store on that initial run. Finally, we would like the robot to be able to 
navigate home following the reverse of this route. It should be noted that this final step is not
strictly accurate to the behaviour of the desert ant. \textit{Wehner et al.} \cite{Wehner2006}
demonstrated that the remembered routes have a distinct polarity, so knowledge of a route from
nest to food, does not imply that the ant has knowledge of a route from food to nest.
\newline

The first stage of the project will focus upon obtaining a working collision avoidance system as a
pre-requisite to gathering the route information. This CA system should be based on visual
information readily available to AntBot with no additional/specialist sensors. For this paper, we
assume that CA is a low-level reactionary behaviour, in that, we don not use any further processing
of the detected motion (e.g. a neural model); we react based on the immediate stimulus of the flow
field. We will look at two different optical flow techniques used to build CA systems. We will also
discuss the effects of using different types of flow field, how the different flow
techniques behave in the same situation, and different methods of response.
\newline

We then move to the Mushroom Body circuit, first estabilishing a baseline performance measure for
the visual navigation task by using the orignal \textit{basic} model from \cite{Ardin2016} with
binary weightings and a single Extrinsic Neuron. A scanning behaviour will be used for this baseline;
ants have demonstrated use of scanning in visual navigation but it is generally accepted that this
is not the primary method they use to determine a direction after having recognised a scene, rather,
this scanning behaviour only occurs in certain scenarios (e.g. when the ant becomes lost)
\cite{Kodzhabashev2006}. 
\newline

Finally, we will report the results of the experiments performed at different stages during, and
post development; we will compare these to relevant results from previous iterations of
this project. We will end with a conclusion of our findings and contributions to the project,
as well as discussing technical limitations and potential for future developments.
\newline

\subsection { Results }
This work is based on work done previously by Leonard Eberding, Luca Scimeca, and Zhaoyu Zhang
\cite{Eberding2016, Scimeca2017, Zhang2017}.
\newline

Significant contributions:
\begin{enumerate}
  \item{An optical flow based system for Collision Avoidance}
  \item{Results indicating the impracticality of an expansion based system for Collision Avoidance}
  \item{\textit{[FUTURE]} Successful replication of a route through a cluttered environment using Visual Navigation}
  \item{\textit{[FUTURE]} Comparison of different Visual Navigation models in a set navigational task}
    
\end{enumerate}


\vfill
\textit{Note: results marked [FUTURE] are results I would like to have achieved by the culmination
of the project and the writing of the final dissertation. They are not current.}
\textit{To be continued...}


\newpage

%%%%%%%%%%%%%%%%%%%%%%%%%%%%%%%%%%%%%%%%%%%%%%%%%%%%%%%%%%%%%%%%%%%%%%%%%%%%%%% BACKGROUND
\section{ Background}
\subsection{ Optical flow models for Collision Avoidance }
\subsubsection{ Expansion }

\subsubsection{ Filtering }

\subsection{ The Mushroom Body for Visual Navigation } \label{MBBackground}

\newpage

%%%%%%%%%%%%%%%%%%%%%%%%%%%%%%%%%%%%%%%%%%%%%%%%%%%%%%%%%%%%%%%%%%%%%%%%%%%%%%% PLATFORM
\section{ Platform }
a

\newpage
%%%%%%%%%%%%%%%%%%%%%%%%%%%%%%%%%%%%%%%%%%%%%%%%%%%%%%%%%%%%%%%%%%%%%%%%%%%%%%% 


\end{document}